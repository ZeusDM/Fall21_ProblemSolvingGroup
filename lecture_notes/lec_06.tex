\lecture{6}{2021-10-20}{}

\begin{prob}{}{}
	Can you show how to express any positive fraction as a sum of distinct positive reciprocal whole numbers? For example, \(7/3 = 1/1 + 1/2 + 1/3 + 1/4 + 1/5 + 1/20\).
\end{prob}

\begin{sol}{}{}
	Let's divide the solution into two lemmas.

	\begin{lem*}{}{}
		If \(\frac{p}{q} < 1\), then there exists a finite subset \(I \subset \mathbb{Z}_{>0}\) so that \[
			\frac{p}{q} = \sum_{i \in I} \frac{1}{i}.
		\]
	\end{lem*}

	\begin{proof}
		Induction on \(p\). If \(p = 0\), set \(I = \varnothing\). If \(p = 1\), set \(I = \{q\}\).

		Supoose that \(p > 1\), and that, for all rational numbers \(\frac{p'}{q'}\) with  \(p' < p\), the statement is true. 

		Let \(n\) be the smallest integer so that \[
			np - q \geq 0.
		\]
		By minimality of \(n\), \((n-1)p - q < 0\), thus  \(np - q < p\).

		Therefore, by the induction hypothesis, there exists a finite subset \(I' \subset \mathbb{Z}_{>0}\) so that \[
			\frac{np - q}{qn} = \sum_{i \in I'} \frac{1}{i}.
		\]

		Notice that, since \(np - q < p < q\), \(\frac{np - q}{qn} < \frac{1}{n}\), so \(n \notin I'\). Therefore, define \(I = I' \cup \{n\}\), and we have
		\begin{align*}
			\frac{p}{q} &= \frac{pn - q}{qn} + \frac{1}{n} \\
						&= \sum_{i \in I'}\frac{1}{i} + \frac{1}{n} \\
						&= \sum_{i \in I}\frac{1}{i},
		\end{align*}
		as desired.
	\end{proof}

	\begin{lem*}{}{}
		Any fraction \(\frac{p}{q}\) can be written as a sum of distinct positive reciprocal of integers.
	\end{lem*}

	\begin{proof}
		Let \(k\) be the largest integer so that \[
			x \leq \frac{1}{1} + \frac{1}{2} + \dots + \frac{1}{k}.
		\] (Since the harmonic series tends to infinity, \(k\) is well-defined.)

		Let \(x' = x - \frac{1}{1} - \frac{1}{2} - \dots - \frac{1}{k}\). By maximality of \(k\), \(x' < \frac{1}{k+1}\). 

		From the previous lemma, since \(x' < \frac{1}{k+1} < 1\), we know that there exists a finite set \(I' \subset \mathbb{Z}_{>0}\) so that \[
			x' = \sum_{i \in I'} \frac{1}{i}.
		\]
		
		Since \(x' < \frac{1}{k+1}\), we conclude that \(1, 2, \dots, k\) are not elements of \(I'\). Define  \(I = \{1, 2, \dots, k\} \cup I'\). Then, we have
		\begin{align*}
			x &= \frac{1}{1} + \frac{1}{2} + \dots + \frac{1}{k} + x' \\
			  &= \sum_{i \in \{1, 2, \dots, k\}} \frac{1}{i} + \sum_{i \in I'} \frac{1}{i} \\
			  &= \sum_{i \in I}\frac{1}{i},
		\end{align*}
		as desired.
	\end{proof}
\end{sol}

