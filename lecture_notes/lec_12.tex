\lecture{12}{2021-12-01}{}

\begin{prob}{Putnam 2010, A2}{}
	Find all differentiable functions $f:\mathbb{R}\to\mathbb{R}$ such that
	\[f'(x)=\frac{f(x+n)-f(x)}n\]
	for all real numbers $x$ and all positive integers $n.$
\end{prob}
\begin{sol}{}{}
	I claim that the only functions \(f\) that satisfy the requirements are \(f(x) = ax + b\), with real parameters \(a, b\).
	It is easy to check that those functions indeed satisfy the requirements; both RHS and LHS are always \(a\).

	Now, suppose \(f\) is a function that satisfies the requirements.
	Let \(g(x) = f(x) - f(0) - xf'(0)\). Then, \(g'(x) = f'(x) - f'(0)\).

	\begin{lem}{}{putnam2010:a2:lem1}
	\[g'(x)=\frac{g(x+n)-g(x)}n\]
	for all real numbers $x$ and all positive integers $n.$
	\end{lem}
	\begin{proof}
	Note that
	\begin{align*}
		\frac{g(x + n) - g(n)}{n} &= \frac{f(x + n) - f(x) - nf'(0)}{n} \\
								  &= \frac{f(x + n) - f(x)}{n} - f'(0) \\
								  &= f'(x) - f'(0) \\
								  &= g'(x),
	\end{align*}
	for all real numbers \(x\) and all positive integers \(n\).
	\end{proof}

	\begin{lem}{}{putnam2010:a2:lem2}
		\[
			g'(x) = g'(x+1)
		\]
		for all real numbers \(x\).
	\end{lem}
	\begin{proof}
		Note that \[
			g'(x) = \frac{g(x+2) - g(x)}{2}
		\]
		and \[
			g'(x) = \frac{g(x+1) - g(x)}{1},
		\]
		therefore we have \[
			g'(x) = \frac{(g(x + 2) - g(x)) - (g(x+1) - g(x))}{2 - 1} = \frac{g(x+2) - g(x+1)}{1} = g'(x+1).
		\]
	\end{proof}

	Since \(g'(0) = 0\), Lemma 2 implies that \(g'(n) = 0\) for all integers \(n\). Then, with Lemma 1, we conclude that \(g(n) = g(0) = 0\) for all integers \(n\).

	\begin{lem}{}{putnam2010:a2:lem3}
		\[
			g(x) = g(x+1)
		\]
		for all real numbers \(x\).
	\end{lem}
	\begin{proof}
		\begin{align*}
			g(x+1) - g(x) &= \int_x^{x+1} g(t)\,dt \\
						  &=  \int_x^{\lfloor x \rfloor} g(t)\,dt + \int_{\lfloor x \rfloor}^{x+1} g(t)\,dt \\
						  &=  \int_x^{\lfloor x \rfloor} g(t)\,dt + \int_{\lfloor x \rfloor - 1}^{x} g(t)\,dt & \text{by Lemma 2}\\
						  &=  \int_{\lfloor x \rfloor - 1}^{\lfloor x \rfloor} g(t)\,dt\\
						  &= g(\lfloor x\rfloor) - g(\lfloor x \rfloor - 1) = 0.
		\end{align*}
	\end{proof}

	Finally, Lemma 1 with \(n = 1\), together with Lemma 3, imply that \[
		g'(x) = 0
	\] for all real numbers \(x\);
	since \(g(0) = 0\), we have that \[
		g(x) = 0
	\]
	for all real numbers \(x\), and consequently \[
		f(x) = f'(0)x + f(0),
	\] which is in the format we desired.
	

\end{sol}
\begin{prob}{Putnam 2011, B1}{}
	Let $h$ and $k$ be positive integers. Prove that for every $\varepsilon >0,$ there are positive integers $m$ and $n$ such that \[\varepsilon < \left|h\sqrt{m}-k\sqrt{n}\right|<2\varepsilon.\]
\end{prob}
\begin{sol}{}{}
	\begin{lem}{}{}
	Let $\epsilon >0$. There are positive integers $m_0$ and $n_0$ such that \[0 < \sqrt{m_0}-\sqrt{n_0} < \epsilon.\]
	\end{lem}
	\begin{proof}
		Let \(m_0 = n_0 + 1\). Let \(n_0\) be large enough so that \(n_0 > \epsilon^{-2}\). Then, \[
			0 < \sqrt{n_0 + 1} - \sqrt{n_0} = \frac{1}{\sqrt{n_0+1} + \sqrt{n_0}} < n_0^{-1/2} < \epsilon.
		\]
	\end{proof}
	\begin{lem}{}{}
	Let $\epsilon >0$. There are positive integers $m_1$ and $n_1$ such that \[\epsilon < \sqrt{m_1}-\sqrt{n_1} < 2\epsilon.\]
	\end{lem}
	\begin{proof}
		Consider \(m_0\) and \(n_0\) from Lemma 1. Let \(\delta = \sqrt{m_0} - \sqrt{n_0}\). We know that \(0 < \delta < \epsilon\). Let  \(\ell = \left\lfloor\frac{\epsilon}{\delta}\right\rfloor + 1\). Then, we know that \[
			\frac{\epsilon}{\delta} < \ell \leq \frac{\epsilon}{\delta} + 1,
		\]
		i.e.
		\[
			\epsilon < \ell \delta \leq \epsilon + \delta < 2\epsilon.
		\]

		Define \(m_1 = \ell^2 m_0\) and \(n_1 = \ell^2 n_0\). Then, \(\sqrt{m_1} - \sqrt{n_1} = \ell\delta\), which is between \(\epsilon\) and \(2\epsilon\), as desired.
	\end{proof}
	\begin{lem}{a.k.a., the problem}{}
	Let $h$ and $k$ be positive integers. Let \(\varepsilon > 0\). There are positive integers $m$ and $n$ such that \[\varepsilon < h\sqrt{m}-k\sqrt{n}<2\varepsilon.\]
	\end{lem}
	\begin{proof}
		Lemma 2 with \(\epsilon \mapsto \frac{\varepsilon}{hk}\) implies that there exist \(m_1\) and \(n_1\) such that \[
			\frac{\varepsilon}{hk} < \sqrt{m_1} - \sqrt{n_1} < \frac{2\varepsilon}{hk}.
		\]
		Define \(m = k^2m_1\) and  \(n = h^2n_1\). Then, \( \varepsilon < h\sqrt{m} - k\sqrt{n} < 2\varepsilon\), as desired.
	\end{proof}
\end{sol}

\begin{prob}{Putnam 2012, B1}{}
	Let $S$ be a class of functions from $[0,\infty)$ to $[0,\infty)$ that satisfies:
	\begin{enumerate}[label = (\roman*), left = 0pt]
		\item The functions $f_1(x)=e^x-1$ and $f_2(x)=\ln(x+1)$ are in $S;$
		\item If $f(x)$ and $g(x)$ are in $S,$ the functions $f(x)+g(x)$ and $f(g(x))$ are in $S;$
		\item If $f(x)$ and $g(x)$ are in $S$ and $f(x)\ge g(x)$ for all $x\ge 0,$ then the function $f(x)-g(x)$ is in $S.$
	\end{enumerate}
	Prove that if $f(x)$ and $g(x)$ are in $S,$ then the function $f(x)g(x)$ is also in $S.$
\end{prob}
\begin{sol}{}{}
	Suppose \(f(x)\) anf \(g(x)\) are in \(S\).

	Then, by (i) and (ii), \[
		\ln(f(x)g(x) + f(x) + g(x) + 1) = 
		\ln(f(x) + 1) + \ln(g(x) + 1) \in S.
	\]

	Then, by (i) and (ii), \[
		f(x)g(x) + f(x) + g(x) = e^{\ln(f(x)g(x) + f(x) + g(x) + 1)} - 1 \in S.
	\]

	Also, by (ii), \(f(x) + g(x) \in S\). Note that \[
		f(x)g(x) + f(x) + g(x) \geq f(x) + g(x)
	\] for all \(x\). Therefore, by (iii), the function \[
	\left(f(x)g(x) + f(x) + g(x)\right) - \left(f(x) + g(x)\right) = f(x)g(x)
	\] is in \(S\), as desired.
\end{sol}

\newpage
\begin{prob}{Putnam 2011, A1}{}
	Define a \textit{growing spiral} in the plane to be a sequence of points with integer coordinates $P_0=(0,0),P_1,\dots,P_n$ such that $n\ge 2$ and:

	\begin{enumerate}[label = \textbullet, left = 0pt]
		\item The directed line segments $P_0P_1$, $P_1P_2$, $\dots$, $P_{n-1}P_n$ are in successive coordinate directions east (for $P_0P_1$), north, west, south, east, etc.
		\item The lengths of these line segments are positive and strictly increasing.
	\end{enumerate}

	\begin{center}
	\vspace{-.5em}
	\setlength{\unitlength}{.4mm}	
	\begin{picture}(200,182)

	\put(20,100){\line(1,0){160}}
	\put(100,10){\line(0,1){170}}

	\put(0,97){\footnotesize West}
	\put(180,97){\footnotesize East}
	\put(90,0){\footnotesize South}
	\put(90,182){\footnotesize North}

	\put(100,100){\circle{1}}\put(100,100){\circle{2}}\put(100,100){\circle{3}}
	\put(115,100){\circle{1}}\put(115,100){\circle{2}}\put(115,100){\circle{3}}
	\put(115,130){\circle{1}}\put(115,130){\circle{2}}\put(115,130){\circle{3}}
	\put(40,130){\circle{1}}\put(40,130){\circle{2}}\put(40,130){\circle{3}}
	\put(40,20){\circle{1}}\put(40,20){\circle{2}}\put(40,20){\circle{3}}
	\put(170,20){\circle{1}}\put(170,20){\circle{2}}\put(170,20){\circle{3}}

	\multiput(100,99.5)(0,.5){3}{\line(1,0){15}}
	\multiput(114.5,100)(.5,0){3}{\line(0,1){30}}
	\multiput(40,129.5)(0,.5){3}{\line(1,0){75}}
	\multiput(39.5,20)(.5,0){3}{\line(0,1){110}}
	\multiput(40,19.5)(0,.5){3}{\line(1,0){130}}

	\put(102,90){\footnotesize \(P_0\)}
	\put(117,90){\footnotesize \(P_1\)}
	\put(117,132){\footnotesize \(P_2\)}
	\put(28,132){\footnotesize \(P_3\)}
	\put(28,10){\footnotesize \(P_4\)}
	\put(172,10){\footnotesize \(P_5\)}
	\end{picture}
	\end{center}
	How many of the points $(x,y)$ with integer coordinates $0\le x\le 2011,0\le y\le 2011$ \textit{cannot} be the last point, $P_n,$ of any growing spiral?
\end{prob}
\begin{sol}{UNFINISHED SOLUTION}{}
	If a growing spiral has \(n = 2\), say it first moves \(a_1\) units east and then \(a_2\) units north; with necessarily \(0 < a_1 < a_2\). Then, the last point \(P_2\) will be \((a_1, a_2)\). In other words, the set of possible last points for a growing spiral with \(n=2\) is \[
		\{(x, y) : 0 < x < y\}.
	\]

	If a growing spiral has \(n = 3\), say it moves \(a_1\) units east, then \(a_2\) units north, then \(a_3\) units west; with necessarily \(0 < a_1 < a_2 < a_3\).
	Then, the last point \(P_3\) will be \((a_1 - a_3, a_2)\).
	Since \(a_1, a_2, a_3\) are integers, the inequalities imply that  \(a_1 - a_3 \leq -2\) and  \(a_2 \geq 2\); so the last point \(P_3\) must be in the set \(\{(x, y) : x \leq -2, y \geq 2\}\).
	Note that, by setting \(a_1 = y - 1 < a_2 = y < a_3 = y - 1 + (-x)\), we obtain any point in such set.
	In other words, the set of possible last points for a growing spiral with \(n = 3\) is \[
		\{(x, y) : x \leq -2, y \geq 2\}.
	\]

	By similar arguments, the set of possible last points for a growing spiral with \(n = 4\) is \[
		\{(x, y) : x \leq -2, y \leq -2\},
	\]
	and the set of possible last points for a growing spiral with \(n = 5\) is 
\end{sol}
