\lecture{2}{2021-09-15}{}

\begin{prob}[Putnam 1999, A1]
Find polynomials $f(x)$, $g(x)$, and $h(x)$, if they exist, such that for all $x$,
\[
|f(x)|-|g(x)|+h(x) = \begin{cases} -1 & \mbox{if $x<-1$} \\
                     3x+2 & \mbox{if $-1 \leq x \leq 0$} \\
                     -2x+2 & \mbox{if $x>0$.}
                     \end{cases}
\]
\end{prob}

\begin{sol}
	The polynomials
	\begin{gather*}
		f(x) = \frac{1}{2} ((3x + 2) - (-1)) = \frac{3}{2}x + \frac{3}{2},\\
		g(x) = \frac{1}{2} ((-2x+2) - (3x+2)) = \frac{5}{2}x, \\
		h(x) = -x + \frac{3}{2}
	\end{gather*}
	satisfy the requirement.
\end{sol}

\newpage
\begin{prob}[Putnam 1999, B2]
	Let $P(x)$ be a polynomial of degree $n$ such that $P(x)=Q(x)P''(x)$, where $Q(x)$ is a quadratic polynomial and $P''(x)$ is the second derivative of $P(x)$.
	Show that if $P(x)$ has at least two distinct roots then it must have $n$ distinct roots.	
\end{prob}

\begin{sol}
	If \(n \leq 2\), then it always holds that, if \(P(x)\) has at least to distinct roots, then it has at least \(n\) distinct roots. Suppose that \(n > 3\).

	We'll equivalently prove that, if \(P(x)\) has a root of multiplicity at least \(2\), then it has a root with multiplicity \(n\).

	In other words, suppose \((x - \alpha)^2 \mid P(x)\). We will show that \((x - \alpha)^n \mid P(x)\).

	Throughout the solution, we'll use the following theorems.
	\begin{lem}
		If \((x - \alpha)^k \mid P(x)\), then \((x-\alpha)^{k-1} \mid P'(x)\).
	\end{lem}
	\begin{lem}
		If \((x-\alpha)\) divides \(P(x)\), \(P'(x)\),  \(\dots\), \(P^{(k-1)}(x)\), then \((x-\alpha)^k \mid P(x)\).
	\end{lem}

	First, if we compare the leading coefficient in the expression \[
		P(x) = Q(x)P''(x),
	\]
	then we conclude the leading coefficient of \(Q(x)\) is  \(\frac{1}{n(n-1)}\).

	Suppose  \(Q(x) \neq \frac{1}{n(n-1)}(x-\alpha)^2\). Then, \((x-\alpha)^2 \mid P(x) = Q(x) P''(x) \implies (x - \alpha) \mid P''(x)\), since the two factors \(x - \alpha\) cannot be both in \(Q(x)\). By the first lemma, \((x - \alpha) \mid P'(x)\). By the second lemma, \((x - \alpha)^3 \mid P(x)\).

	We'll prove, using induction, that \((x-\alpha)^k \mid P(x)\) for any positive integer \(k\).

	Suppose \((x-\alpha)^k \mid P(x) = Q(x) P''(x) \implies (x - \alpha)^{k-1} \mid P''(x)\), since the two factors \(x - \alpha\) cannot be both in \(Q(x)\). By the first lemma, \((x - \alpha)^{k-1} \mid P''(x) \implies (x - \alpha)^{k-2} \mid P^{(3)}(x) \implies \cdots \implies (x - \alpha) \mid P^{(k)}(x)\). By the second lemma, \((x - \alpha)^{k+1} \mid P(x)\); which finishes the induction.

	This implies that \(P(x)\) has a root with multiplicity \(n+1\), which contradicts the fact that the degree of \(P(x)\) is \(n\).
	Therefore, \(Q(x) = \frac{1}{n(n-1)} (x-\alpha)^2\).

	Let's differentiate the original equation twice:
	\begin{gather*}
		P'(x) = Q(x)P^{(3)}(x) + Q'(x)P''(x)\\
		P''(x) = Q(x)P^{(4)}(x) + 2Q'(x)P^{(3)}(x) + Q''(x)P''(x)
	\end{gather*}
	Notice that \((x-\alpha)\) divides \(Q(x)P^{(4)}(x) + 2Q'(x)P^{(3)}(x)\), therefore, it also must divide \(P''(x)(1 - Q''(x)) = P''(x)\left(1 - \frac{1}{\binom{n}{2}}\right)\). Since \(1 - \frac{1}{\binom{n}{2}} \neq 0\), we conclude \((x-\alpha)\) divides \(P''(x)\).

	In general, using that \(Q^{(3)}(x) = 0\), we have \[
		P^{(k)} = Q(x)P^{(k+2)}(x) + kQ'(x)P^{(k+1)}(x) + \binom{k}{2}Q''(x)P^{(k)}(x).
	\]

	So we similarly conclude \((x-\alpha)\) divides \(\left(1 - \frac{\binom{k}{2}}{\binom{n}{2}}\right)P^{(k)}\), and, as long as \(k \neq n\), we conclude that \((x-\alpha)\) divides \(P^{(k)}(x)\). Thus, by the second lemma, \((x-\alpha)^n \mid P(x)\), as desired.
\end{sol}

\newpage
\begin{prob}[Putnam 2014, A1]
	Prove that every nonzero coefficient of the Taylor series of \[
		(1 - x + x^2)e^x
	\] about $x=0$ is a rational number whose numerator (in lowest terms) is either $1$ or a prime number.
\end{prob}

\begin{sol}
	Since \[
		e^x = \sum_{n=0}^\infty \frac{x^n}{n!},
	\]
	we conclude that
	\begin{align*}
		(1 - x - x^2)e^x &= \sum_{n=0}^\infty \frac{x^n}{n!} - \sum_{n=0}^\infty \frac{x^{n+1}}{n!} +  \sum_{n=0}^\infty \frac{x^{n+2}}{n!} \\
		&= \sum_{n=0}^\infty \frac{x^n}{n!} - \sum_{n=1}^\infty \frac{x^{n}}{(n-1)!} +  \sum_{n=2}^\infty \frac{x^{n}}{(n-2)!} \\
		&= 1 + \sum_{n=2}^\infty x^n\left( \frac{1}{n!} - \frac{1}{(n-1)!} + \frac{1}{(n-2)!} \right) \\
		&= 1 + \sum_{n=2}^\infty x^n \frac{1 - n + n(n-1)}{n!} \\
		&= 1 + \sum_{n=2}^\infty x^n \frac{(n-1)^2}{n!}\\
		&= 1 + \sum_{n=2}^\infty x^n \frac{(n-1)}{(n-2)! \cdot n}
	\end{align*}

	If \(n - 1\) is prime, we're good. If \(n - 1 = 4\), then \(\frac{4}{3!\cdot5} = \frac{2}{15}\), so we're good. If \(n - 1 = p^2\), with \(p > 2\), then \(n-1 = p^2 \mid p \cdot (2p) \mid (n-2)!\), so the numerator of the fraction is \(1\). Otherwise, we can find \(n - 1 > a > b > 1\) such that  \(n - 1 = ab\), therefore  \(n-1 = ab \mid (n-2)!\) so the numeratior is \(1\).
\end{sol}
