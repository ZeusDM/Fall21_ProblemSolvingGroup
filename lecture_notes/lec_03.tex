\lecture{3}{2021-09-22}{}

\begin{prob}{}{}
	Can three points with integer coornidates in the plane be vertices of an equilateral triangle? What about in three dimentions?
\end{prob}

\begin{sol}{}{}
	Let \(ABC\) be an equilateral triangle in the plane so that \(A = (a_x, a_y) \in \mathbb{Z}^2\), \(B = (b_x, b_y) \in \mathbb{Z}^2\), and \(C = (c_x, c_y)\), with \(A\), \(B\), and \(C\) being distinct. Without loss of generality, suppose \(A\), \(B\), \(C\) are in counterclockwise order. Then,
	\begin{align*}
		(C - A) &=
		\begin{pmatrix}
			\cos(\pi/3) & -\sin(\pi/3) \\
			\sin(\pi/3) & \cos(\pi/3) 
		\end{pmatrix}
		(B - A) \\
		&=
		\begin{pmatrix}
			1/2 & -\sqrt{3}/2 \\
			\sqrt{3}{2} & 1/2 
		\end{pmatrix}
		\begin{pmatrix}
			b_x - a_x \\
			b_y - a_y
		\end{pmatrix} \\
		&=
		\begin{pmatrix}
			1/2(b_x - a_x) - \sqrt{3}/{2}(b_y-a_y) \\
			\sqrt{3}/2(b_x - a_x) + 1/2(b_y-a_y) \\
		\end{pmatrix}.
	\end{align*}

	Finally, since \(A \neq B\), \(b_x - a_x \neq 0\) or \(b_y - b_x \neq 0\), which implies \(C - A \notin \mathbb{Q}^2 \implies C \notin \mathbb{Q}^2 \implies C \notin \mathbb{Z}^2\). Therefore, there is no equilateral triangle in the plane with vertices with integer coordinates.

	In three dimensions, the points \((1, 0, 0)\), \((0, 1, 0)\), and \((0, 0, 1)\) form an equilateral triangle.
\end{sol}

\begin{sol}{using areas}{}
	Let's use the same notation as above, and suppose \(A, B, C \in \mathbb{Z}^2\).
	On one hand, using the same notation as above, the area of the triangle \(ABC\) is the absolute value of the determinant of \[
		\frac{1}{2}\begin{pmatrix}
			a_x&a_y&1 \\
			b_x&b_y&1 \\
			c_x&c_y&1
		\end{pmatrix},
	\]
	which is a rational number.
	On the other hand, the area of an equilateral triangle is \[
		\frac{\ell^2 \sqrt{3}}{2},
	\]
	where \(\ell\) is the length of the side. Pythagoras' theorem implies that \(\ell^2 = |AB|^2\) is an integer, so the area of \(ABC\) is an irrational number; a contradiction. Thus, no such triangle exists.
\end{sol}
