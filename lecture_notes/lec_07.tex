\lecture{7}{2021-10-27}{}
\begin{prob}{Putnam 2015, A2}{}
	Given a list of the positive integers $1,2,3,4,\dots,$ take the first three numbers $1,2,3$ and their sum $6$ and cross all four numbers off the list. Repeat with the three smallest remaining numbers $4,5,7$ and their sum $16.$ Continue in this way, crossing off the three smallest remaining numbers and their sum and consider the sequence of sums produced: $6,16,27, 36, \dots.$ Prove or disprove that there is some number in this sequence whose base 10 representation ends with $2015.$
\end{prob}

\begin{sk}{}{}
	\textcolor{red}{The full solution is to be done.} The answer is yes. You can prove, using induction, that the \((k+1)\)-th term of the sequence will be \(10k + s_k\), with \(s_k \in \{5, 6, 7\}\). You will be able to recursevely evaluate in terms of \(s_q\), you write \(k = 3q + r\), \(0 \leq r < 3\).

	With that, you will be able to find \(k\) such that \(k \equiv 201 \pmod 1000\) and \(s_k = 5\).
\end{sk}

