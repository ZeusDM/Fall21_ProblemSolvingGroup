\lecture{4}{2021-09-29}{}

\begin{prob}
	When \(4444^{4444}\) is written in decimal notation, the sum of its digits is \(A\).  Let \(B\) be the sum of the digits of \(A\). Find the sum of the digits of \(B\).
\end{prob}

\begin{sol}
	Let \(C\) be the sum of the digits of \(B\).

	First, we will investigate the size of the numbers.
	Since \(0 < 4444^{4444} < 10000^{4444} = 10^{4\cdot4444}\), we conclude \(0 < A \leq 9 \cdot 4 \cdot 4444\). Since \(0 < A < 10^6\), \(0 < B \leq 9 \cdot 6 = 36\). Finally, this implies \(0 < C \leq 2 + 9 = 11\).

	If we write any number \(n\) in its decimal representation, i.e., \(n = a_0 + a_110 + a_210^2 + \cdots + a_k10^k\), then we conlude \[
		n = a_0 + a_110 + a_210^2 + \cdots + a_k10^k \equiv a_0 + a_1 + a_2 + \cdots + a_k \pmod{9}.
	\]

	Therefore, \(7 \equiv 4444^{4444} \equiv A \equiv B \equiv C \pmod{9}\). Since \(0 < C \leq 11\), \(C\) must be \(7\).
\end{sol}

\begin{prob}%[Putnam 2003, B3]
	Show that for each positive integer $n$, \[
		n! = \prod_{i=1}^n  \mathrm{lcm}\{1, 2, \dots, \lfloor n/i\rfloor\} .
	\] (Here $\mathrm{lcm}$ denotes the least common multiple, and
	$\lfloor x \rfloor$ denotes the greatest integer $\leq x$.)
\end{prob}

\begin{sol}
	\textcolor{red}{Solution to be written.}
\end{sol}
