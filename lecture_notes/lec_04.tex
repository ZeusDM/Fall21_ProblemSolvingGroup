\lecture{4}{2021-09-29}{}

%\begin{prob}
	%When \(4444^{4444}\) is written in decimal notation, the sum of its digits is \(A\).  Let \(B\) be the sum of the digits of \(A\). Find the sum of the digits of \(B\).
%\end{prob}
%
%\begin{prob}
	%Evaluate \[ \sin\left(\frac{\pi}{11}\right) \sin\left(\frac{2\pi}{11}\right) \cdots \sin\left(\frac{10\pi}{11}\right) \]exactly.
%\end{prob}
%
%\begin{prob}
	%Find the $2000$\textsuperscript{th} digit in the square root of $N = 11\dots1$, where $N$ contains $1998$ digits, all of them $1$'s.
%\end{prob}
%
%\begin{prob}
	%Can you show how to express any positive fraction as a sum of distinct positive reciprocal whole numbers? For example, $7/3 = 1/1 + 1/2 + 1/3 + 1/4 + 1/5 + 1/20$.
%\end{prob}
%
%\begin{prob}
	%Can the portion of any parabola inside a circle of radius \(1\) have a length greater than \(4\)?
%\end{prob}

%\begin{prob}%[Putnam 2003, B3]
	%Show that for each positive integer $n$,
	%\[
		%n! = \prod_{i=1}^n  \mathrm{lcm}\{1, 2, \dots, \lfloor n/i\rfloor\} .
	%\]
	%(Here $\mathrm{lcm}$ denotes the least common multiple, and
	%$\lfloor x \rfloor$ denotes the greatest integer $\leq x$.)
%\end{prob}
