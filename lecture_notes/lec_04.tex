\lecture{4}{2021-09-29}{}

\begin{prob}{IMO 1975, 4}{}
	When \(4444^{4444}\) is written in decimal notation, the sum of its digits is \(A\).  Let \(B\) be the sum of the digits of \(A\). Find the sum of the digits of \(B\).
\end{prob}

\begin{sol}{}{}
	Let \(C\) be the sum of the digits of \(B\).

	First, we will investigate the size of the numbers.
	Since \(0 < 4444^{4444} < 10000^{4444} = 10^{4\cdot4444}\), we conclude \(0 < A \leq 9 \cdot 4 \cdot 4444\). Since \(0 < A < 10^6\), \(0 < B \leq 9 \cdot 6 = 36\). Finally, this implies \(0 < C \leq 2 + 9 = 11\).

	If we write any number \(n\) in its decimal representation, i.e., \(n = a_0 + a_110 + a_210^2 + \cdots + a_k10^k\), then we conlude \[
		n = a_0 + a_110 + a_210^2 + \cdots + a_k10^k \equiv a_0 + a_1 + a_2 + \cdots + a_k \pmod{9}.
	\]

	Therefore, \(7 \equiv 4444^{4444} \equiv A \equiv B \equiv C \pmod{9}\). Since \(0 < C \leq 11\), \(C\) must be \(7\).
\end{sol}

\begin{prob}{Putnam 2003, B3}{}
	Show that for each positive integer $n$, \[
		n! = \prod_{i=1}^n  \mathrm{lcm}\{1, 2, \dots, \lfloor n/i\rfloor\} .
	\] (Here $\mathrm{lcm}$ denotes the least common multiple, and
	$\lfloor x \rfloor$ denotes the greatest integer $\leq x$.)
\end{prob}

\begin{sk}{UNFINISHED SOLUTION}{}
	Let \(\nu_p(n)\) be the largest \(\alpha\) so that \(p^\alpha \mid n\).\footnote{See \href{https://en.wikipedia.org/wiki/P-adic_order}{\(p\)-adic order on Wikipedia}.}
	In order to show that LHS equals RHS, it suffices to show that \(\nu_p(\mathrm{LHS}) = \nu_p(\mathrm{RHS})\) for all primes \(p\).

	We know that\footnote{This is known as \href{https://en.wikipedia.org/wiki/Legendre's_formula}{Legendre's formula}.}
	\begin{align*}
		\nu_p\left(n!\right) &= \sum_{i=1}^\infty \left\lfloor\frac{n}{p^i}\right\rfloor\\
		&= \sum_{i=1}^n \left\lfloor\frac{n}{p^i}\right\rfloor.
	\end{align*}

	We also know that
	\begin{align*}
		\nu_p\left(\prod_{i=1}^n \mathrm{lcm}\{1, 2, \dots, \lfloor n/i \rfloor\} \right)
		&= \sum_{i=1}^n \nu_p\left(\mathrm{lcm}\{1, 2, \dots, \lfloor n/i \rfloor\} \right).
	\end{align*}


	Finally, recall that \(\nu_p(\mathrm{lcm}\{1, 2, \dots, \lfloor n/i \rfloor\})\) is the largest number \(\alpha\) so that \[
		p^\alpha \mid \mathrm{lcm}\{1, 2, \dots, \lfloor n/i \rfloor\},
	\] which means that, if we set \(\alpha = \nu_p(\mathrm{lcm}\{1, 2, \dots, \lfloor n/i \rfloor\})\), it holds that \[
		p^\alpha \mid \mathrm{lcm}\{1, 2, \dots, \lfloor n/i \rfloor\}
	\] and \[
		p^{\alpha+1} \nmid \mathrm{lcm}\{1, 2, \dots, \lfloor n/i \rfloor\}.
	\]

	By definition of least common multiple, we conlcude \(p^\alpha \mid k\) for some \(1 \leq k \leq \lfloor n/i \rfloor\) and \(p^{\alpha+1} \nmid k\) for all  \(1 \leq k \leq \lfloor n/i \rfloor\).

	By definition of floor, the statement above is equivalent to 
	\(p^\alpha \mid k\) for some \(1 \leq k \leq n/i\) and \(p^{\alpha+1} \nmid k\) for all  \(1 \leq k \leq n/i\).

	Again, the statement above implies that \(1 \leq p^\alpha \leq n/i\), while \(n/i < p^{\alpha+1}\), i.e., \(\alpha\) is the unique number satisfying \[
		\frac{n}{p} < ip^{\alpha} \leq n
	\]
\end{sk}
