\documentclass{scrartcl}
\KOMAoptions{
    fontsize=12pt,
    paper=letter,
    paper=portrait,
    parskip=half,
    headings=big,
    toc=listof,
    twoside=false,
	DIV=14,
    }

\usepackage{amsmath, amsthm, graphicx, enumitem}

\pagestyle{empty}

\theoremstyle{definition}
\newtheorem{prob}{Problem}

\begin{document}
	\begin{minipage}{.6\textwidth}{}
		\textsc{Haverford Problem Solving Group}

		\textsc{October 27, 2021}
	\end{minipage}
	\begin{minipage}{.4\textwidth}{}
		\ \hfill
		\includegraphics[height = .9in]{psg_logo}
	\end{minipage}\\[1em]

	\hrule

	\setcounter{prob}{2}
	\begin{prob}
		Find the $2000$\textsuperscript{th} digit in the square root of $N = 11\dots1$, where $N$ contains $1998$ digits, all of them $1$'s.
	\end{prob}

	\setcounter{prob}{5}
	\begin{prob}
		Can the portion of any parabola inside a circle of radius \(1\) have a length greater than \(4\)?
	\end{prob}

	\setcounter{prob}{9}

	\begin{minipage}{.85\textwidth}{}
		\begin{prob}%[Pick's Theorem]
			Suppose that a polygon has integer coordinates for all of its vertices. Let \(i\) be the number of integer points that are interior to the polygon, and let \(b\) be the number of integer points on its boundary (including vertices as well as points along the sides of the polygon). Then the area of this polygon is \[
				i + \frac{b}{2} - 1.
			\]
		\end{prob}
	\end{minipage}
	\begin{minipage}{.15\textwidth}{}
		\ \hfill
		\includegraphics[width=.85\textwidth]{pick}
	\end{minipage}

	\begin{prob}%[Romanian Masters of Mathematics 2018, 2]
		Determine whether there exist non-constant polynomials $P\left(x\right)$ and $Q\left(x\right)$ with real coefficients satisfying \[P\left(x\right)^{10}+P\left(x\right)^9=Q\left(x\right)^{21}+Q\left(x\right)^{20}.\]
	\end{prob}

	\begin{prob}%[Romanian Masters of Mathematics 2018, 3]
		Ann and Bob play a game on an infinite checkered plane making moves in turn. A move consists in orienting any unit grid-segment that has not been oriented before. If at some stage some oriented segments form an oriented cycle, Bob wins.
		\begin{enumerate}[label = (\alph*)]
			\item Bob makes the first move. Does Bob have a strategy that guarantees him to win?
			\item Ann makes the first move. Does Bob have a strategy that guarantees him to win?
		\end{enumerate}
	\end{prob}

	\begin{prob}%[Putnam 2015, B2]
		Given a list of the positive integers $1,2,3,4,\dots,$ take the first three numbers $1,2,3$ and their sum $6$ and cross all four numbers off the list. Repeat with the three smallest remaining numbers $4,5,7$ and their sum $16.$ Continue in this way, crossing off the three smallest remaining numbers and their sum and consider the sequence of sums produced: $6,16,27, 36, \dots.$ Prove or disprove that there is some number in this sequence whose base 10 representation ends with $2015.$
	\end{prob}

	\begin{prob}%[Putnam 2014, B2]
		Suppose that $f$ is a function on the interval $[1,3]$ such that $-1\le f(x)\le 1$ for all $x$ and $\displaystyle \int_1^3f(x)\,dx=0.$ How large can $\displaystyle\int_1^3\frac{f(x)}x\,dx$ be?
	\end{prob}

	\vfill

	\begin{minipage}{.85\textwidth}{}
		\footnotesize
		If you are not in our Discord server, you should definitely join.
		We will post there handouts, resources, solutions, room/time changes, and (most important of all) pictures whatever food we will have in the meeting. Point your phone camera to the QR code to join it.
	\end{minipage}
	\begin{minipage}{.15\textwidth}{}
		\ \hfill \includegraphics[height = .8in]{qr}
	\end{minipage}
\end{document}
