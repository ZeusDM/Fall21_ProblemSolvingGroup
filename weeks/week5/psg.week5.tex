\documentclass{scrartcl}
\KOMAoptions{
    fontsize=12pt,
    paper=letter,
    paper=portrait,
    parskip=half,
    headings=big,
    toc=listof,
    twoside=false,
	DIV=14,
    }

\usepackage{amsmath, amsthm, graphicx}

\pagestyle{empty}

\theoremstyle{definition}
\newtheorem{prob}{Problem}

\begin{document}
	\begin{minipage}{.6\textwidth}{}
		\textsc{Haverford Problem Solving Group}

		\textsc{October 6, 2021}
	\end{minipage}
	\begin{minipage}{.4\textwidth}{}
		\ \hfill
		\includegraphics[height = .9in]{psg_logo}
	\end{minipage}\\[1em]

	\hrule

	\setcounter{prob}{1}
	\begin{prob}[Modified after last week's progress]
		Prove that \[ \sin\left(\frac{\pi}{11}\right) \sin\left(\frac{2\pi}{11}\right) \cdots \sin\left(\frac{10\pi}{11}\right) = \frac{11}{2^{10}},\]
		or more generally, prove that \[ \sin\left(\frac{\pi}{n}\right) \sin\left(\frac{2\pi}{n}\right) \cdots \sin\left(\frac{(n-1)\pi}{n}\right) = \frac{n}{2^{n-1}}.\]
	\end{prob}

	\begin{prob}
		Find the $2000$\textsuperscript{th} digit in the square root of $N = 11\dots1$, where $N$ contains $1998$ digits, all of them $1$'s.
	\end{prob}

	\setcounter{prob}{4}
	\begin{prob}
		Can you show how to express any positive fraction as a sum of distinct positive reciprocal whole numbers? For example, $7/3 = 1/1 + 1/2 + 1/3 + 1/4 + 1/5 + 1/20$.
	\end{prob}

	\begin{prob}
		Can the portion of any parabola inside a circle of radius \(1\) have a length greater than \(4\)?
	\end{prob}

	\setcounter{prob}{7}
	\begin{prob}%[Putnam 2004, A3]
		Define a sequence \((u_n)_{n=0}^\infty\) by  \(u_0 = u_1 = u_2 = 1\) and thereafter by the condition that  \[
			\det
			\begin{pmatrix}
				u_n & u_{n+1} \\
				u_{n+2} & u_{n+3}
			\end{pmatrix} = n!
		\] for all \(n \geq 0\). Show that \(u_n\) is an integer for all \(n\). 
		(By convention, \(0! = 1\).)
	\end{prob}

	\begin{prob}%[Putnam 2005, A1]
		Show that every positive integer is a sum of one or more numbers of the form \(2^r3^s\), where \(r\) and \(s\) are nonnegative integers and no summand divides another. (For example, \(23 = 9 + 8 + 6\).)
	\end{prob}

	\vfill

	\begin{minipage}{.85\textwidth}{}
		\footnotesize
		If you are not in our Discord server, you should definitely join.
		We will post there handouts, resources, solutions, room/time changes, and (most important of all) pictures whatever food we will have in the meeting. Point you phone camera to the QR code to join it.
	\end{minipage}
	\begin{minipage}{.15\textwidth}{}
		\ \hfill \includegraphics[height = .8in]{qr}
	\end{minipage}
\end{document}
