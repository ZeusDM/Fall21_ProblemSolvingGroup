\documentclass[20pt]{beamer}
\geometry{papersize={8.5in,11in}, top = 0in, bottom = 0in, right = .5in, left = .5in}

\setbeamertemplate{navigation symbols}{}
%\usefonttheme{serif}

%\setbeamertemplate{background}{\tikz[overlay,remember picture]\node[opacity=.25]at (current page.center){\includegraphics[width=10in]{background_light}};}
\setbeamertemplate{background}{\tikz[overlay,remember picture]\node[opacity=1]at (current page.center){\includegraphics[width=10in]{background_dark}};}

\usepackage{tikz}

\usepackage{graphicx}

\newcommand\poster[2]{
\begin{frame}
	\begin{center}
		{\Huge \textbf{#1}}

		\vfill

		{\LARGE #2}

		\vfill

		Join the \textbf{Problem Solving Group.} \\ 
		We work together to do math, real math, for fun;\\
		with pizza, cake, or cookies. \\[1em]

		Scan below to join our Discord server or contact \texttt{gdantasemo@haverford.edu} to get more info.

		\vspace{.3in}

		\includegraphics[width=1.5in]{qr}\hspace{1in}\includegraphics[width = 1.5in]{psg_logo}
	\end{center}
\end{frame}
}

\begin{document}
\color{white}
	\poster{Intrigued?}{When \(4444^{4444}\) is written in decimal notation, the sum of its digits is \(A\).  Let \(B\) be the sum of the digits of \(A\). Find the sum of the digits of \(B\).}
	\poster{Good With Numbers?}{Evaluate \[
			\mathrm{\sin\left(\frac{\pi}{11}\right)
			\sin\left(\frac{2\pi}{11}\right)
			\cdots
		\sin\left(\frac{10\pi}{11}\right)}
	\]exactly.}
	\poster{Like Math?}{Can three points with integer coornidates in the plane be vertices of an equilateral triangle? What about in three dimentions?}
	\poster{Enjoy Puzzles?}{Can you show how to express any positive fraction as a sum of distinct positive reciprocal whole numbers?

		\vspace{.3in}

	{\normalsize For example, \\ $7/3 = 1/1 + 1/2 + 1/3 + 1/4 + 1/5 + 1/20$}}
	\poster{Curious?}{Can the portion of any parabola inside a circle of radius \(1\) have a length greater than \(4\)?}
	\poster{Got a Calculator?}{Find the $2000$th digit in the square root of $N = 11\dots1$, where $N$ contains $1998$ digits, all of them $1$’s.}
\end{document}
